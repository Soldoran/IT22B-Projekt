%--------------------------------------------------------
	% Variablen
	\newcommand{\myTitle}{To-Do-App in Blazor}
	\newcommand{\myAuthor}{Leo- Jonathan - Sebastian}
	\newcommand{\myLineThickness}{0.4pt}
	
	% Externe Variablen Einbinden
	%---------------------------------------------------------------------
	% Formatier-Hilfen
		% Einen Absatz ohne Einrückungen machen oder andere Formattierungen
		\newcommand{\neuerAbsatz}{\newline \linebreak \noindent}
%---------------------------------------------------------------------
	% Texte-Abschnitt
	% Hier sind die gesamten Texte des Dokumentes.
	% Die Texte werden je in eigenen Commands(Variablen) gespeichert welche dann im Code benutzt werden können.
	% Syntax: [lowercase(Section-Bezeichnung)+Text]
%---------------------------------------------------------------------

\newcommand{\einleitungText}{
	Die folgende Projektdokumentation beschreibt das Ersatzleistungsprojekt, welches statt einer schriftlichen Klausur in Lernfeld 8 geleistet wird. Das Projekt wurde durch die Gruppe Leo M., Jonathan G. und Sebastian K. im Zeitraum vom 15.05.2024 bis 19.06.2024 absolviert. Die beteiligten an diesem Projekt absolvieren je eine Ausbildung zum Fachinformatiker in den Fachgebieten Digitale Vernetzung und Anwendungsentwicklung im Ausbildungsbetrieb inray Industriesoftware GmbH.
}

\newcommand{\projektumfeldText}{
	Für die Durchführung des Projektes wird von dem Berufsbildungszentrum Rendsburg, folgend BBZ genannt, während des wöchentlichen Unterrichtes für 90 Minuten, ein Arbeitsplatz zur Verfügung gestellt, welcher mit einem Computer mit geeigneter Hardware für die Softwareentwicklung und einer Internetverbindung ausgestattet war. Alternativ durften auch Geräte des Ausbildungsbetriebes oder Privatgeräte zur Entwicklung genutzt werden.\neuerAbsatz Der Auftraggeber des Projektes ist das BBZ, vertreten durch den Klassenlehrer Till Gades.
}

\newcommand{\projektzielText}{
	Ziel des Projektes ist es, eine To-Do Applikation zu entwickeln, mit welcher Nutzer sich Anmelden können und Listen mit Aufgaben erstellen, bearbeiten, als erledigt kennzeichnen, teilen oder löschen können. Die Daten sollen dabei in einer SQL-Datenbank gesichert werden.\neuerAbsatz Die Applikation soll auch das Teilen von Listen oder Aufgaben zwischen Nutzern unterstützen, ähnlich eines Ticketsystems.
}

\newcommand{\projektbegruendungText}{
	Viele To-Do Applikationen versprechen meist wenig Umfang oder sind mit Kosten verbunden. Eine solche To-Do-App zu entwickeln ist nicht nur ein beliebtes Anfängerprojekt in der Objektorientierten Programmierung, sondern lässt sich auch gut skalieren, wie zum Beispiel durch das Managen von mehreren Nutzern oder das Teilen von Listen mit anderen, wodurch die Appliklation auch für die Planung von Gruppenaktivitäten mit mehreren Personen gut geeignet ist.
}

\newcommand{\projektschnittstellenText}{
	% Gibt es nicht
}

\newcommand{\personelleSchnittstellenText}{
	Als Gruppe werden die fachlichen und technischen Anforderungen definiert sowie Mockups der Benutzeroberfläche besprochen. Bei Problemen oder Fragen während des Projektes steht den Autoren auch der Klassenlehrer der IT22B des BBZ zur Verfügung.
}

\newcommand{\technischeSchnittstellenText}{
	Das Projekt ist eine alleinstehende Applikation, welche von der Gruppe in Zusammenarbeit entwickelt wurde. Genauere Erläuterungen zur Funktionsweise und technischen Details werden in folgenden Kapiteln behandelt.
}

\newcommand{\projektabgrenzungText}{
	% Gibt es nicht
}

\newcommand{\projektplanungText}{
	% Gibt es nicht
}

\newcommand{\projektphasenText}{
	Durchgeführt wurde die Projektarbeit in dem Zeitraum vom 15.05.2024 bis zum 19.06.2024 während der regulären Schulzeiten, sowie freiwillig in der Freizeit der Projektteilnehmer. Für das Projekt waren insgesamt 9 Stunden (6 Unterrichtstunden zu je 90 Minuten) Bearbeitungszeit geplant. Die grobe Zeitplanung ist in der Untenstehenden Tabelle 1 zu sehen und die genaue Zeitplanung mit Differenz befindet sich im Anhang A1.
	\ref{tab:tabelle1}
}

\newcommand{\ressourcenplanungText}{
	Alle der notwendigen Ressourcen waren vor dem Projektbeginn bereits vorhanden, wie beispielsweise Notebooks, Internetzuugänge, Entwicklungsumgebung (Visual Studio 2022), Git etc.\neuerAbsatz Der Personalbedarf beschränlt sich auf die Autoren.\neuerAbsatz Zur optimalen Zeit-/Kosten Nutzung werden neben Microsofts Front-End Web Framework Blazor, sowie die öffentlich zugängliche Bibliothek Radzen.Blazor genutzt. Für das Backend wird eine lokale MS SQL Datenbank aufgesetzt.
}

\newcommand{\entwicklungsprozessText}{
	Für den Entwicklungsprozess des Projektes wurde sich für ein Wasserfall-Modell entschieden, da dies das einfachste und effektivste Planungsmodell für ein Projekt in diesem Umfang ist. Eine agile Projektmanagementmethode wie Scrum wurde bewusst nicht gewählt, da das Projekt klare und definierte Anforderungen hat.\neuerAbsatz Scrum bietet Flexibilität bei sich ändernden Anforderungen im Projektverlauf, da iterativ in Sprint gearbeitet wird und somit Änderungen möglich sind.\neuerAbsatz Das Wasserfall-Modell bietet eine klare Struktur mit definierten Phasen, was die Planung und Durchführung mit klaren Zielen erleichtert. Diese klare Struktur unterstützt die Entwicklung der Applikation, da die Anfoderungen für das Projekt fest gegeben sind. In Abbildung 1 im Anhang A2 sieht man die Planung in einem Wasserfall-Modell.
}

\newcommand{\analysephaseText}{
	% Gibt es nicht
}

\newcommand{\istanalyseText}{
	\lipsum[5]
}

\newcommand{\wirtschaftlichkeitsanalyseText}{
	% Gibt es nicht
}

\newcommand{\makeorbuyText}{
	\lipsum[4]
}

\newcommand{\projektkostenText}{
	Die Kosten für die Durchführung des Projekts setzen sich sowohl aus Personal-, als auch aus Ressourcenkosten zusammen. Laut Ausbildungsvertrag verdient ein Auszubildender im zweiten Lehrjar bei inray monatlich 950 € (brutto).
	
	\begin{equation*}
		8 \frac{\text{h}}{\text{Tag}} \times 220 \frac{\text{Tage}}{\text{Jahr}} = 1{,}760 \frac{\text{h}}{\text{Jahr}}
	\end{equation*}
	
	\begin{equation*}
		950 \frac{\text{€}}{\text{Monat}} \times 12 \frac{\text{Monate}}{\text{Jahr}} = 11{,}400 \frac{\text{€}}{\text{Jahr}}
	\end{equation*}
	
	\begin{equation*}
		\frac{11{,}400 \frac{\text{€}}{\text{Jahr}}}{1{,}760 \frac{\text{h}}{\text{Jahr}}} \approx 6{,}48 \frac{\text{€}}{\text{h}}
	\end{equation*}
	\neuerAbsatz		
	Daraus ergibt sich ein Stundensatz von 6,48 EUR je Gruppenmitglied. Die Dauer der Projektdurchführung beträgt 9 Stunden. Für die Nutzung von Ressourcen wird ein pauschaler Stundensatz von 8 EUR angenommen. Die Kosten für das Projekt sind in Tabelle 2 dargestellt und belaufen sich auf insgesamt 390,96 EUR.
}

\newcommand{\amortisationsrechnungText}{
	Zur Berechnung der Amortisation des Projektes wurde eine Schnittpunktanalyse zwischen den Gesamtkosten des Projektes und dem vorraussichtlichen Preis von 20 EUR pro verkaufter Lizenz durchgeführt
	
	
	\begin{align*}
		& \text{Anzahl verkaufter Lizenzen bis Amortisation} = \frac{\text{Gesamtprojektkosten}}{\text{Verkaufspreis pro Lizenz}} \\ \\
		& \text{Anzahl verkaufter Lizenzen bis Amortisation} = \frac{390{,}96 €}{20 €} \\ \\
		& \text{Anzahl verkaufter Lizenzen bis Amortisation} \approx 19{,}5
	\end{align*}
	\neuerAbsatz
	Sobald 20 Lizenzen verkauft werden, amortisiert sich das Projekt.
}

\newcommand{\qualitaetsanforderungText}{
	Die Qualitätsanforderungen für die Ergebnisse dieses Projektes orientieren sich an dern Qualitätskriterien für Software nach ISO/IEC 9126-1 [2001]. 
}

\newcommand{\nichtfinanzielleVorteileText}{
	% Gibt es nicht
}

\newcommand{\entwurfsphaseText}{
	% Gibt es nicht
}

\newcommand{\zielplattformText}{
	% Gibt es nicht
}

\newcommand{\plattformText}{
	Durch das in diesem Projekt genutzte Framework Blazor, entwickelt von Microsoft, lässt sich die Applikation auf so gut wie jedem System Nutzen, da man den Dienst auf einem Server hosten kann und nach belieben im eigenem Netzwerk oder über das Internet mit jedem Browser erreichen kann.
}

\newcommand{\programmierspracheText}{
	Das Blazor-Framework benutzt als Programmiersprache CSharp. Dank bereits vorhandenen Kenntnissen und Fähigkeiten in dieser Sprache, fiel die Wahl der Umsetzung in Blazor schnell und die Gruppe kann von den Erfahrungen aus dem Ausbildungsbetrieb profitieren. Zusätzlich bietet sich so die Möglichkeit, erfahrenere Kollegen bei Problemen befragen zu können.
}

\newcommand{\grafischeOberflaecheText}{
	Bei der To-Do-App handelt es sich um eine lokal installierte Webanwendung, die das Framework Blazor in CSharp zur Darstellung von GUI-Elementen verwendet. Dabei werden Radzen.Blazor- und Bootstrap-Komponenten genutzt, um ein modernes und einheitliches Erscheinungsbild zu gewährleisten.
}

\newcommand{\architekturdesignText}{
	% Gibt es nicht
}

\newcommand{\bibliothekenText}{
	% Text für Entscheidung der Bibliotheken hier einfügen
}

\newcommand{\strukturellesDesignText}{
	% Text für Strukturelles Design hier einfügen
}

\newcommand{\funktionalesDesignText}{
	% Text für Funktionales Design hier einfügen
}

\newcommand{\entwurfbenuzteroberflaecheText}{
	% Text für Entwurf der Benutzeroberfläche hier einfügen
}

\newcommand{\massnahmenQualitaetssicherungText}{
	% Text für Maßnahmen zur Qualitätssicherung hier einfügen
}

\newcommand{\implementierungsphaseText}{
	% Text für Implementierungsphase hier einfügen
}

\newcommand{\implementierungproviderText}{
	% Text für Implementierung der Provider hier einfügen
}

\newcommand{\implementierunglaufzeittransferText}{
	% Text für Implementierung der Laufzeit für Transfer-Objekte hier einfügen
}

\newcommand{\implementierunglaufzeittriggerText}{
	% Text für Implementierung der Laufzeit für den Trigger hier einfügen
}

\newcommand{\implementierungunittestsText}{
	% Text für Implementierung von Unittests hier einfügen
}

\newcommand{\implementierungbenuzteroberflaecheText}{
	% Text für Implementierung der Benutzeroberfläche hier einfügen
}

\newcommand{\abnahmephaseText}{
	% Text für Abnahmephase hier einfügen
}

\newcommand{\dokumentationText}{
	% Text für Dokumentation hier einfügen
}

\newcommand{\fazitText}{
	% Text für Fazit hier einfügen
}

\newcommand{\sollistvergleichText}{
	% Text für Soll-/Ist-Vergleich hier einfügen
}

\newcommand{\reflexionText}{
	% Text für Reflexion hier einfügen
}

\newcommand{\ausblickText}{
	% Text für Ausblick hier einfügen
}

\newcommand{\quellenverzeichnisText}{
	% Text für Quellenverzeichnis hier einfügen
}

\newcommand{\anhangText}{
	% Text für Anhang hier einfügen
}

\newcommand{\anhangEins}{
	% Text für Ausblick im Anhang hier einfügen
}

\newcommand{\anhangZwei}{
	% Text für Ausblick im Anhang hier einfügen
}

\newcommand{\anhangDrei}{
	% Text für Ausblick im Anhang hier einfügen
}
%--------------------------------------------
%--------------------------------------------------------
% Configurations
	\documentclass[ngerman,11pt,a4paper,titlepage]{article}
	
	% Titelseite
	\title{\myTitle}
	\author{\myAuthor}

% Packages
	\usepackage[T1]{fontenc}
	\usepackage{graphicx}
	\usepackage{amssymb}
	\usepackage{amsthm}
	\usepackage{xcolor}
	\usepackage{nameref}
	\usepackage{babel}
	\usepackage{fancyhdr}
	%\usepackage{hyperref}
	\usepackage{lipsum}
	\usepackage{amsmath}
	\usepackage{amssymb}
	\usepackage{caption}

% Header & Footer Einstellungen
	\setlength{\headheight}{14pt}	% Höhe des Headers festlegen
	\setlength{\headsep}{20pt}		% Abstand zwischen Header und Text festlegen
	
	\pagestyle{fancy}
	\fancyhf{}
	\fancyhead[L]{}
	\fancyhead[C]{\myTitle}
	\fancyhead[R]{}
	
	% Text Footer-Sektionen
	\fancyfoot[L]{}
	\fancyfoot[C]{\thepage}
	\fancyfoot[R]{}
	
	% Die Trennlinien bei Header und Footer
	\renewcommand{\headrulewidth}{\myLineThickness}
	\renewcommand{\footrulewidth}{\myLineThickness}
%--------------------------------------------------------


\begin{document}
	\maketitle % Wird verwendet um den Titel + Author anzuzeigen
	\tableofcontents % Erstellt automatisch ein Inhaltverzeichnis aus den (sub)sections
	\clearpage
	
% --------------------------------
	% Anfang Dokumentation
	\section{Einleitung}
	\label{sec:einleitung}
	\einleitungText
	
	\subsection{Projektumfeld}
	\label{subsec:projektumfeld}
	\projektumfeldText
	
	\subsection{Projektziel}
	\label{subsec:projektziel}
	\projektzielText
	
	\subsection{Projektbegründung}
	\label{subsec:projektbegruendung}
	\projektbegruendungText
	
	\subsection{Projektschnittstellen}
	\label{subsec:projektschnittstellen}
	\projektschnittstellenText
	
	\subsubsection{Personelle Schnittstellen}
	\label{subsubsec:personelleschnittstellen}
	\personelleSchnittstellenText
	
	\subsubsection{Technische Schnittstellen}
	\label{subsubsec:technischeschnittstellen}
	\technischeSchnittstellenText
	
	% Dieser Abschnitt macht meiner Meinung keinen Sinn in dieser Doku (- Sebastian)
	%\subsection{Projektabgrenzung}
	%\label{subsec:projektabgrenzung}
	%\projektabgrenzungText
%\pagebreak
	
	\section{Projektplanung}
	\label{sec:projektplanung}
	\projektplanungText
	
	\subsection{Projektphasen}
	\label{subsec:projektphasen}
	\projektphasenText
	\neuerAbsatz
	\begin{tabular}[h]{l|r}
		Projektphase & Geplante Zeit \\
		\hline
		Gruppen- und Projektfindung & 1.5 Stunden \\
		\hline
		Ausarbeitung des Grobkonzeptes & 1.5 Stunden \\
		\hline
		Ausarbeitung der Umsetzungsphase & 1.5 Stunden \\
		\hline
		Durchführen des Umsetzungsphase & 3.0 Stunden \\
		\hline
		Abschließen der Dokumentation und Vorbereitung der Präsentation & 1.5 Stunden \\
		\hline
		Gesamt & 9.0 Stunden \\
		\label{tab:tabelle1}
	\end{tabular}
	
	\subsection{Ressourcenplanung}
	\label{subsec:ressourcenplanung}
	\ressourcenplanungText
	
	\subsection{Entwicklungsprozess}
	\label{subsec:entwicklungsprozess}
	\entwicklungsprozessText
%\pagebreak
	
	\section{Analysephase}
	\label{sec:analysephase}
	\analysephaseText
	
	\subsection{Ist-Analyse}
	\label{subsec:istanalyse}
	\istanalyseText
	
	\subsection{Wirtschaftlichkeitsanalyse}
	\label{subsec:wirtschaftlichkeitsanalyse}
	\wirtschaftlichkeitsanalyseText
	
	\subsubsection{Make or Buy-Entscheidung}
	\label{subsubsec:makeorbuy}
	\makeorbuyText
	
	\subsubsection{Projektkosten}
	\label{subsubsec:projektkosten}
	\projektkostenText
	\neuerAbsatz
	\begin{tabular}{|c|c|c|c|}
		Vorgang & Zeit (h) & Kosten / Stunde & Kosten \\
		\hline
		Entwicklung (pro Person) & 9 & 6,48 € + 8 € = 14,48 € & 130,32 € \\
		\hline
		Gesamt & 27 & 43,44 € & 390,96 €
		\label{tab:tabelle2}
	\end{tabular}
	
	\subsubsection{Amortisationsrechnung}
	\label{subsubsec:amortisationsrechnung}
	\amortisationsrechnungText
	
	\subsection{Qualitätsanforderung}
	\label{subsec:qualitaetsanforderung}
	\qualitaetsanforderungText
	
	% Entfällt, keine Idee was man hier schreiben könnte
	%\subsection{Nicht-finanzielle Vorteile}
	%\label{subsec:nichtfinanzielleVorteile}
	%\nichtfinanzielleVorteileText
%\pagebreak
	
	\section{Entwurfsphase}
	\label{sec:entwurfsphase}
	\entwurfsphaseText
	
	\subsection{Zielplattform}
	\label{subsec:zielplattform}
	\zielplattformText
	
	\subsubsection{Browser}
	\label{subsubsec:plattform}
	\plattformText
	
	\subsubsection{Entscheidung der Programmiersprache}
	\label{subsubsec:programmiersprache}
	\programmierspracheText
	
	\subsubsection{Umsetzung der grafischen Oberfläche}
	\label{subsubsec:grafischeoberflaeche}
	\grafischeOberflaecheText
	
	\subsection{Architekturdesign}
	\label{subsec:architekturdesign}
	\architekturdesignText
	
	\subsubsection{Entscheidung der Bibliotheken}
	\label{subsubsec:bibliotheken}
	\bibliothekenText
	
	\subsubsection{Strukturelles Design}
	\label{subsubsec:strukturellesdesign}
	\strukturellesDesignText
	
	\subsubsection{Funktionales Design}
	\label{subsubsec:funktionalesdesign}
	\funktionalesDesignText
	
	\subsection{Entwurf der Benutzeroberfläche}
	\label{subsec:entwurfbenuzteroberflaeche}
	\entwurfbenuzteroberflaecheText
	
	\subsection{Maßnahmen zur Qualitätssicherung}
	\label{subsec:massnahmenqualitaetssicherung}
	\massnahmenQualitaetssicherungText
%\pagebreak
		
	\section{Implementierungsphase}
	\label{sec:implementierungsphase}
	\implementierungproviderText
	
		\subsection{Implementierung der Laufzeit für Transfer-Objekte}
		\label{subsec:implementierunglaufzeittransfer}
		\implementierunglaufzeittransferText
		
		\subsection{Implementierung der Laufzeit für den Trigger}
		\label{subsec:implementierunglaufzeittrigger}
		\implementierunglaufzeittriggerText
		
		\subsection{Implementierung von Unittests}
		\label{subsec:implementierungunittests}
		\implementierungunittestsText
		
		\subsection{Implementierung der Benutzeroberfläche}
		\label{subsec:implementierungbenuzteroberflaeche}
		\implementierungbenuzteroberflaecheText
%\pagebreak
	
	\section{Abnahmephase}
	\label{sec
	}
	\abnahmephaseText
%\pagebreak
	
	\section{Dokumentation}
	\label{sec
	}
	\dokumentationText
%\pagebreak
	
	\section{Fazit}
	\label{sec
	}
	\fazitText
		\subsection{Soll-/Ist-Vergleich}
		\label{subsec:sollistvergleich}
		\sollistvergleichText
		
		\subsection{Reflexion}
		\label{subsec:reflexion}
		\reflexionText
		
		\subsection{Ausblick}
		\label{subsec:ausblick}
		\ausblickText
%\pagebreak
	
	\section{Quellenverzeichnis}
	\label{sec
	}
	\quellenverzeichnisText
%\pagebreak
	
	\section{Anhang}
	\label{sec
	}
	\anhangText
	\subsection{A1}
	\label{subsec:a1}
	\anhangEins
	
	\subsection{A2}
	\label{subsec:a2}
	\anhangZwei
	
	\subsection{A3}
	\label{subsec:a3}
	\anhangDrei
	\pagebreak
	
	% Ende Dokumentation
% ---------------------------------
\end{document}
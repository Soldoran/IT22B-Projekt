\newcommand{\einleitungText}{
	Die folgende Projektdokumentation beschreibt das Ersatzleistungsprojekt, welches statt einer schriftlichen Klausur in Lernfeld 8 geleistet wird. Das Projekt wurde durch die Gruppe Leo M., Jonathan G. und Sebastian K. im Zeitraum vom 15.05.2024 bis 19.06.2024 absolviert. Die beteiligten an diesem Projekt absolvieren je eine Ausbildung zum Fachinformatiker in den Fachgebieten Digitale Vernetzung und Anwendungsentwicklung im Ausbildungsbetrieb inray Industriesoftware GmbH.
}

\newcommand{\projektumfeldText}{
	Für die Durchführung des Projektes wird von dem Berufsbildungszentrum Rendsburg, folgend BBZ genannt, während des wöchentlichen Unterrichtes für 90 Minuten, ein Arbeitsplatz zur Verfügung gestellt, welcher mit einem Computer mit geeigneter Hardware für die Softwareentwicklung und einer Internetverbindung ausgestattet war. Alternativ durften auch Geräte des Ausbildungsbetriebes oder Privatgeräte zur Entwicklung genutzt werden.\neuerAbsatz Der Auftraggeber des Projektes ist das BBZ, vertreten durch den Klassenlehrer Till Gades.
}

\newcommand{\projektzielText}{
	Ziel des Projektes ist es, eine To-Do Applikation zu entwickeln, mit welcher Nutzer sich Anmelden können und Listen mit Aufgaben erstellen, bearbeiten, als erledigt kennzeichnen, teilen oder löschen können. Die Daten sollen dabei in einer SQL-Datenbank gesichert werden.\neuerAbsatz Die Applikation soll auch das Teilen von Listen oder Aufgaben zwischen Nutzern unterstützen, ähnlich eines Ticketsystems.
}

\newcommand{\projektbegruendungText}{
	Viele To-Do Applikationen versprechen meist wenig Umfang oder sind mit Kosten verbunden. Eine solche To-Do-App zu entwickeln ist nicht nur ein beliebtes Anfängerprojekt in der Objektorientierten Programmierung, sondern lässt sich auch gut skalieren, wie zum Beispiel durch das Managen von mehreren Nutzern oder das Teilen von Listen mit anderen, wodurch die Appliklation auch für die Planung von Gruppenaktivitäten mit mehreren Personen gut geeignet ist.
}

\newcommand{\projektschnittstellenText}{
	% Gibt es nicht
}

\newcommand{\personelleSchnittstellenText}{
	Als Gruppe werden die fachlichen und technischen Anforderungen definiert sowie Mockups der Benutzeroberfläche besprochen. Bei Problemen oder Fragen während des Projektes steht den Autoren auch der Klassenlehrer der IT22B des BBZ zur Verfügung.
}

\newcommand{\technischeSchnittstellenText}{
	Das Projekt ist eine alleinstehende Applikation, welche von der Gruppe in Zusammenarbeit entwickelt wurde. Genauere Erläuterungen zur Funktionsweise und technischen Details werden in folgenden Kapiteln behandelt.
}

\newcommand{\projektabgrenzungText}{
	% Bewusst nicht Implementiert (- Sebastian)
}
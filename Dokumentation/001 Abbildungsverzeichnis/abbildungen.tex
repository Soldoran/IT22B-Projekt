% Sicherstellen dass im Hauptodkument folgendes Packet enthalten ist:
%\usepackage{graphicx}

% Das Abbildungsverzeichnis wird im Hauptdokument durch den Befehl \listoffigures erstellt. Dabei sind die Captions an den figures wichtig!

% Wie man eine 'figure' erstellt:
% \begin{figure}
	% \includegraphics{Dateiname.Dateiende} <- Optional
	% \caption{Abbildungsbeschriftung}
% \end{figure}

% Für Einträge im Abbildungsverzeichnis, die sich von der Abbildungsbeschriftung unterscheiden sollen, gibt es für \caption ein alternatives Argument:

% \begin{figure}
% 	...
% 	\caption[Eintrag im Abbildungsverzeichnis]{Abbildungsbeschriftung}
% \end{figure}

% Mit dem Befehl "\addtocontents" lassen sich auch Einträge manuell hinzufügen. Dies gilt auch für Tabellen. Hier wird dann analog statt der figure-Umgebung die table-Umgebung verwendet.

% Source: https://www.heise.de/tipps-tricks/LaTeX-Abbildungsverzeichnis-automatisch-erstellen-7261837.html

% -------------------------------------------------------
% Template Figure
\newcommand{\abbildung}{
	\begin{figure}
		\includegraphics{../img/figure001.png}
		\caption{Abbildung 1: Testimage}
	\end{figure}
}

% -------------------------------------------------------

% -------------------------------------------------------
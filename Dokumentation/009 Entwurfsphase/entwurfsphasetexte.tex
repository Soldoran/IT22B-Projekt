\newcommand{\entwurfsphaseText}{
	% Gibt es nicht
}

\newcommand{\zielplattformText}{
	% Gibt es nicht
}

\newcommand{\plattformText}{
	Durch das in diesem Projekt genutzte Framework Blazor, entwickelt von Microsoft, lässt sich die Applikation auf so gut wie jedem System Nutzen, da man den Dienst auf einem Server hosten kann und nach belieben im eigenem Netzwerk oder über das Internet mit jedem Browser erreichen kann.
}

\newcommand{\programmierspracheText}{
	Das Blazor-Framework benutzt als Programmiersprache CSharp. Dank bereits vorhandenen Kenntnissen und Fähigkeiten in dieser Sprache, fiel die Wahl der Umsetzung in Blazor schnell und die Gruppe kann von den Erfahrungen aus dem Ausbildungsbetrieb profitieren. Zusätzlich bietet sich so die Möglichkeit, erfahrenere Kollegen bei Problemen befragen zu können.
}

\newcommand{\grafischeOberflaecheText}{
	Bei der To-Do-App handelt es sich um eine lokal installierte Webanwendung, die das Framework Blazor in CSharp zur Darstellung von GUI-Elementen verwendet. Dabei werden Radzen.Blazor- und Bootstrap-Komponenten genutzt, um ein modernes und einheitliches Erscheinungsbild zu gewährleisten.
}

\newcommand{\architekturdesignText}{
	% Gibt es nicht
}

\newcommand{\bibliothekenText}{
	\lipsum[1]
}

\newcommand{\strukturellesDesignText}{
	\lipsum[1]
}

\newcommand{\funktionalesDesignText}{
	\lipsum[1]
}

\newcommand{\entwurfbenuzteroberflaecheText}{
	\lipsum[1]
}

\newcommand{\massnahmenQualitaetssicherungText}{
	\lipsum[1]
}
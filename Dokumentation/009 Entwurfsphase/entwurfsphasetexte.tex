\newcommand{\entwurfsphaseText}{
	% Gibt es nicht
}

\newcommand{\zielplattformText}{
	% Gibt es nicht
}

\newcommand{\plattformText}{
	Durch das in diesem Projekt genutzte Framework Blazor, entwickelt von Microsoft, lässt sich die Applikation auf so gut wie jedem System Nutzen, da man den Dienst auf einem Server hosten kann und nach belieben im eigenem Netzwerk oder über das Internet mit jedem Browser erreichen kann.
}

\newcommand{\programmierspracheText}{
	Das Blazor-Framework benutzt als Programmiersprache C\#. Dank bereits vorhandenen Kenntnissen und Fähigkeiten in dieser Sprache, fiel die Wahl der Umsetzung in Blazor schnell und die Gruppe kann von den Erfahrungen aus dem Ausbildungsbetrieb profitieren. Zusätzlich bietet sich so die Möglichkeit, erfahrenere Kollegen bei Problemen befragen zu können.
}

\newcommand{\grafischeOberflaecheText}{
	Bei der To-Do-App handelt es sich um eine lokal installierte Webanwendung, die das Framework Blazor in C\# zur Darstellung von GUI-Elementen verwendet. Dabei werden Radzen.Blazor- und Bootstrap-Komponenten genutzt, um ein modernes und einheitliches Erscheinungsbild zu gewährleisten.
}

\newcommand{\architekturdesignText}{
	% Gibt es nicht
}

\newcommand{\bibliothekenText}{
	Aufgrund der bisherigen Erfahrungen des Teams, durch die im Ausbildungsbetriebes genutzten Bibliotheken, wurde sich für die Komponentenbibliothek \colorbox{lightgray}{Blazor.Radzen} der Firma Radzen entschieden. Diese bietet unter der MIT-Lizenz diverse Komponenten zur Nutzung in Blazor-Projekten an. Diese sind für die Umsetzung des Projektes zwar nicht nötig, da Blazor von Haus aus bereits viele Komponenten anbietet und man auch eigene Komponenten, maßgeschneidert an die eigenen Bedürfnisse und die des Teams, erstellen kann, aber durch den Zeitlich begrenzten Faktor und durch eine ausfürhliche Dokumentation zu den Radzen-Komponenten, fiel die Wahl auf die Blazor.Radzen Komponentenbibliothek.
}

\newcommand{\strukturellesDesignText}{
	Um ein strukturelles Design zu entwerfen und die Planung fortzuführen, hat sich Sebastian K. zunächst mit der Dokumentation und der Funktionsweise der Bibliotheken vertraut gemacht und überprüft, ob die gewollten Funktionen unterstützt werden.
}

\newcommand{\entwurfbenuzteroberflaecheText}{
	Zur Gestaltung der Benutzeroberfläche wurde zunächst ein Mockup erstellt (siehe Anhang 11, Abbildung 11). 
}

\newcommand{\massnahmenQualitaetssicherungText}{
	Um die Funktionsweise und somit die Qualität des Projektes sicherzustellen, wurde am Ende des Zeitraumes zur Bearbeitung bereits früh Zeit eingeplant, sodass die geforderten Funktionen ausgiebig getestet werden können.
}
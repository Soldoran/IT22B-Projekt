\newcommand{\analysephaseText}{
	% Gibt es nicht
}

\newcommand{\istanalyseText}{
	\lipsum[1]
}

\newcommand{\wirtschaftlichkeitsanalyseText}{
	% Gibt es nicht
}

\newcommand{\makeorbuyText}{
	\lipsum[1]
}

\newcommand{\projektkostenText}{
	Die Kosten für die Durchführung des Projekts setzen sich sowohl aus Personal-, als auch aus Ressourcenkosten zusammen. Laut Ausbildungsvertrag verdient ein Auszubildender im zweiten Lehrjar bei inray monatlich 950 € (brutto).
	
	\begin{equation*}
		8 \frac{\text{h}}{\text{Tag}} \times 220 \frac{\text{Tage}}{\text{Jahr}} = 1{,}760 \frac{\text{h}}{\text{Jahr}}
	\end{equation*}
	
	\begin{equation*}
		950 \frac{\text{€}}{\text{Monat}} \times 12 \frac{\text{Monate}}{\text{Jahr}} = 11{,}400 \frac{\text{€}}{\text{Jahr}}
	\end{equation*}
	
	\begin{equation*}
		\frac{11{,}400 \frac{\text{€}}{\text{Jahr}}}{1{,}760 \frac{\text{h}}{\text{Jahr}}} \approx 6{,}48 \frac{\text{€}}{\text{h}}
	\end{equation*}
	\neuerAbsatz		
	Daraus ergibt sich ein Stundensatz von 6,48 EUR je Gruppenmitglied. Die Dauer der Projektdurchführung beträgt 9 Stunden. Für die Nutzung von Ressourcen wird ein pauschaler Stundensatz von 8 EUR angenommen. Die Kosten für das Projekt sind in Tabelle 2 dargestellt und belaufen sich auf insgesamt 390,96 EUR.
}

\newcommand{\amortisationsrechnungText}{
	Zur Berechnung der Amortisation des Projektes wurde eine Schnittpunktanalyse zwischen den Gesamtkosten des Projektes und dem vorraussichtlichen Preis von 20 EUR pro verkaufter Lizenz durchgeführt
	
	
	\begin{align*}
		& \text{Anzahl verkaufter Lizenzen bis Amortisation} = \frac{\text{Gesamtprojektkosten}}{\text{Verkaufspreis pro Lizenz}} \\ \\
		& \text{Anzahl verkaufter Lizenzen bis Amortisation} = \frac{390{,}96 €}{20 €} \\ \\
		& \text{Anzahl verkaufter Lizenzen bis Amortisation} \approx 19{,}5
	\end{align*}
	\neuerAbsatz
	Sobald 20 Lizenzen verkauft werden, amortisiert sich das Projekt.
}

\newcommand{\qualitaetsanforderungText}{
	Die Qualitätsanforderungen für die Ergebnisse dieses Projektes orientieren sich an dern Qualitätskriterien für Software nach ISO/IEC 9126-1 [2001]. 
}

\newcommand{\nichtfinanzielleVorteileText}{
	% Gibt es nicht
}
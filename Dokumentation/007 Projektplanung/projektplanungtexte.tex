\newcommand{\projektplanungText}{
	% Gibt es nicht
}

\newcommand{\projektphasenText}{
	Durchgeführt wurde die Projektarbeit in dem Zeitraum vom 15.05.2024 bis zum 19.06.2024 während der regulären Schulzeiten, sowie freiwillig in der Freizeit der Projektteilnehmer. Für das Projekt waren insgesamt 9 Stunden (6 Unterrichtstunden zu je 90 Minuten) Bearbeitungszeit geplant. Die grobe Zeitplanung ist in der Untenstehenden Tabelle 1 zu sehen und die genaue Zeitplanung mit Differenz befindet sich im Anhang A1.
	\ref{tab:tabelle1}
}

\newcommand{\ressourcenplanungText}{
	Alle der notwendigen Ressourcen waren vor dem Projektbeginn bereits vorhanden, wie beispielsweise Notebooks, Internetzugänge, Entwicklungsumgebung (Visual Studio 2022), Git etc.\neuerAbsatz Der Personalbedarf beschränlt sich auf die Autoren.\neuerAbsatz Zur optimalen Zeit-/Kosten Nutzung werden neben Microsofts Front-End Web Framework Blazor, sowie die öffentlich zugängliche Bibliothek Radzen.Blazor genutzt. Für das Backend wird eine lokale MS SQL Datenbank aufgesetzt.
}

\newcommand{\entwicklungsprozessText}{
	Für den Entwicklungsprozess des Projektes wurde sich für ein Wasserfall-Modell entschieden, da dies das einfachste und effektivste Planungsmodell für ein Projekt in diesem Umfang ist. Eine agile Projektmanagementmethode wie Scrum wurde bewusst nicht gewählt, da das Projekt klare und definierte Anforderungen hat.\neuerAbsatz Scrum bietet Flexibilität bei sich ändernden Anforderungen im Projektverlauf, da iterativ in Sprints gearbeitet wird und somit Änderungen möglich sind.\neuerAbsatz Das Wasserfall-Modell bietet eine klare Struktur mit definierten Phasen, was die Planung und Durchführung mit klaren Zielen erleichtert. Diese klare Struktur unterstützt die Entwicklung der Applikation, da die Anfoderungen für das Projekt fest gegeben sind. In Abbildung 1 im Anhang A1 sieht man die Planung in einem Wasserfall-Modell.
}
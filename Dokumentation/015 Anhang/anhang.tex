\newcommand{\anhang}
{
	\section{Anhang}
	\label{sec:anhang}
	\anhangText
	
	\subsection{A1 Wasserfall-Modell}
	\label{subsec:a1}
	\paragraph{Wasserfall-Modell}
	\anhangEins
	\clearpage
	
	\subsection{A2 To-Dino Applikation}
	\label{subsec:a2}
	\anhangZwei
	\clearpage
	
	\subsection{A3 ApplicationDbContext.cs}
	\label{subsec:a3}
	\anhangDrei
	\clearpage
	
	\subsection{A4 Datenbank Init File}
	\label{subsec:a4}
	\anhangVier
	
	\subsection{A5 Datenbank Migration Snapshot}
	\label{subsec:a5}
	\anhangFuenf
	
	\subsection{A6 LoginModal.cs}
	\label{subsec:a6}
	\anhangSechs
	
	\subsection{A7 ToDoItem.cs}
	\label{subsec:a7}
	\anhangSieben
	
	\subsection{A8 ToDoItemModal.cs}
	\label{subsec:a8}
	\anhangDrei
	
	\subsection{A9 ToDoListComponent.cs}
	\label{subsec:a9}
	\anhangNeun
	
	\subsection{A10 Userclass.cs}
	\label{subsec:a10}
	\anhangZehn
	
	\subsection{A11 To-Dino Mockup}
	\label{subsec:a11}
	\anhangElf
	
	\subsection{A12 ToDoItem Modaldialog Mockup}
	\label{subsec:a12}
	\anhangZwoelf
}
\newcommand{\glossar}{
\newglossaryentry{<glossarkopf>}{name={Begriff}, description={Beschreibung / Ergänzung}}

\newglossaryentry{<todinobeschreibung>}{name={To-Dino}, description={To-Dino ist eine für Webbrowser Optimierte To-Do-Listen Applikation zum erstellen und verwalten von To-Do Listen}}

\newglossaryentry{<radzenbeschreibung>}{name={Radzen}, description={Trademark der Firma Radzen}}

\newglossaryentry{<gitbeschreibung>}{name={Git}, description={Git ist ein Tool zur Versionskontrolle in Softwareprojekten}}

\newglossaryentry{<csharpbeschreibung>}{name={C\#}, description={CSharp (abgekürzt C\#) ist eine objektorientierte Programmiersprache}}

\newglossaryentry{<blazorbeschreibung>}{name={Blazor}, description={Blazor ist ein Webframework welches das Entwickeln von Webapplikationen in C\# und HTML ermöglicht}}

\newglossaryentry{<visualstudiobeschreibung>}{name={Visual Studio}, description={Entwicklungsumgebung von Microsoft}}

\newglossaryentry{<corereviewbeschreibung>}{name={Core-Review}, description={Manuelle Überprüfung von Programmcode, um Fehler zu verhindern und das Einhalten von Standards zu garantierenäufiger Entwurf einer geplanten Benutzeroberfläche}}

\newglossaryentry{<mockupbeschreibung>}{name={Mockup}, description={Vorläufiger Entwurf einer geplanten Benutzeroberfläche}}

\newglossaryentry{<debuggerbeschreibung>}{name={Debugger}, description={Software-Tool, welches Entwicklern hilft Code eines Programmes zu analysieren und Fehler zu finden}}
}